\documentclass[12pt,a4paper]{article}
\usepackage[italian]{babel}
\usepackage[utf8]{inputenc}
\usepackage{amsmath, amssymb}
\usepackage{graphicx}
\usepackage{booktabs}
\usepackage{hyperref}
\usepackage{geometry}
\geometry{margin=2.5cm}

\title{Analisi dei prezzi dell'elettricità in funzione dell'irradiazione solare}
\author{}
\date{}

\begin{document}

\maketitle

\section{Introduzione}
L’obiettivo di questa analisi è verificare la relazione tra il prezzo dell’energia elettrica e l’irradiazione solare giornaliera in Italia.  
L’ipotesi di partenza è che un aumento dell’irradiazione solare (e quindi della produzione fotovoltaica potenziale) tenda a ridurre il prezzo dell’elettricità sul mercato all’ingrosso, a causa di una maggiore offerta di energia.

\section{Raccolta dei dati}
I dati sono stati ottenuti da due fonti distinte:

\begin{itemize}
    \item \textbf{Yahoo Finance}: tramite la libreria \texttt{yfinance} di Python sono stati scaricati i prezzi giornalieri di un future sull’elettricità europea (ticker \texttt{ELEC.F}). Poiché Yahoo Finance non fornisce direttamente il prezzo unico nazionale italiano (PUN), tale future è stato utilizzato come \emph{proxy} rappresentativo dei movimenti di prezzo nel mercato elettrico europeo.
    \item \textbf{NASA POWER API}: i dati meteorologici giornalieri di irradiazione solare (\emph{Global Horizontal Irradiance}, GHI, espressa in kWh/m$^2$/day) sono stati ottenuti tramite l’API pubblica del progetto NASA POWER, per le coordinate geografiche di Padova (latitudine 45.406, longitudine 11.876). È stato utilizzato il parametro \texttt{ALLSKY\_SFC\_SW\_DWN}, relativo all’energia solare incidente sulla superficie terrestre.
\end{itemize}

Entrambe le serie temporali coprono l’intervallo dal 1 gennaio al 31 dicembre 2024.

\section{Elaborazione dei dati}
Le due serie sono state importate in Python sotto forma di \texttt{DataFrame} di \texttt{pandas}.  
Successivamente:
\begin{enumerate}
    \item sono state uniformate le date al formato \texttt{YYYY-MM-DD};
    \item è stato eseguito un \texttt{merge} sulle date comuni, ottenendo un dataset unico con tre colonne principali: data, prezzo e irradiazione solare;
    \item sono stati rimossi eventuali valori mancanti.
\end{enumerate}

A partire dal dataset risultante, è stato stimato un modello di regressione lineare ordinaria (OLS) della forma:
\begin{equation}
    \text{Prezzo}_t = \beta_0 + \beta_1 \cdot \text{Irradiazione}_t + \varepsilon_t
\end{equation}
dove:
\begin{itemize}
    \item $\text{Prezzo}_t$ rappresenta il prezzo giornaliero dell’elettricità nel giorno $t$;
    \item $\text{Irradiazione}_t$ è l’irradiazione solare media giornaliera;
    \item $\varepsilon_t$ è il termine di errore aleatorio.
\end{itemize}

L’analisi è stata condotta mediante il pacchetto \texttt{statsmodels}, che fornisce le stime dei coefficienti $\beta_0$, $\beta_1$, l’errore standard, il valore di $R^2$ e i test di significatività statistica.

\section{Visualizzazione}
Per una rappresentazione grafica della relazione è stato generato un diagramma di dispersione (\emph{scatter plot}) tra prezzo e irradiazione solare, accompagnato dalla retta di regressione stimata:
\begin{center}
% \includegraphics[width=0.8\textwidth]{plot_regressione.pdf}
\end{center}
La linea di regressione è calcolata come:
\[
    \hat{y}_t = \hat{\beta}_0 + \hat{\beta}_1 \cdot \text{Irradiazione}_t
\]
dove $\hat{y}_t$ rappresenta il prezzo stimato.

\section{Risultati e interpretazione}
Dall’analisi emerge una relazione (in genere) di tipo negativo tra irradiazione solare e prezzo dell’elettricità.  
In termini economici, ciò significa che nei giorni con maggiore irradiazione — e quindi maggiore produzione fotovoltaica — il prezzo dell’energia tende a ridursi per effetto dell’aumento dell’offerta.

Il coefficiente $\hat{\beta}_1$ misura la variazione media del prezzo per unità aggiuntiva di irradiazione solare (kWh/m$^2$/day).  
Un valore negativo di $\hat{\beta}_1$ conferma l’ipotesi iniziale, mentre un valore positivo indicherebbe un effetto opposto o l’influenza di altri fattori di mercato (es. domanda, idroelettrico, termico).

Il coefficiente di determinazione $R^2$ indica quanto della variabilità del prezzo è spiegata dall’irradiazione solare.  
Poiché il prezzo dell’elettricità dipende da molte altre variabili (domanda, meteo, costo del gas, congestioni di rete, ecc.), ci si aspetta un valore di $R^2$ moderato.

\section{Conclusioni}
Questo semplice modello OLS mostra come i dati di mercato e quelli meteorologici possano essere combinati per analizzare le determinanti del prezzo dell’energia elettrica.  
Nonostante la semplicità, il metodo fornisce un’evidenza quantitativa del legame tra produzione rinnovabile e prezzi di mercato.

Un’estensione naturale dell’analisi includerebbe:
\begin{itemize}
    \item l’aggiunta di variabili esplicative (temperatura, domanda elettrica, prezzo del gas);
    \item la distinzione stagionale tramite \emph{dummy} mensili;
    \item l’uso di modelli dinamici o a componenti multiple (es. regressione multipla o modelli ARIMAX).
\end{itemize}

\end{document}
