\documentclass[11pt]{article}

    \usepackage[breakable]{tcolorbox}
    \usepackage{parskip} % Stop auto-indenting (to mimic markdown behaviour)
    

    % Basic figure setup, for now with no caption control since it's done
    % automatically by Pandoc (which extracts ![](path) syntax from Markdown).
    \usepackage{graphicx}
    % Keep aspect ratio if custom image width or height is specified
    \setkeys{Gin}{keepaspectratio}
    % Maintain compatibility with old templates. Remove in nbconvert 6.0
    \let\Oldincludegraphics\includegraphics
    % Ensure that by default, figures have no caption (until we provide a
    % proper Figure object with a Caption API and a way to capture that
    % in the conversion process - todo).
    \usepackage{caption}
    \DeclareCaptionFormat{nocaption}{}
    \captionsetup{format=nocaption,aboveskip=0pt,belowskip=0pt}

    \usepackage{float}
    \floatplacement{figure}{H} % forces figures to be placed at the correct location
    \usepackage{xcolor} % Allow colors to be defined
    \usepackage{enumerate} % Needed for markdown enumerations to work
    \usepackage{geometry} % Used to adjust the document margins
    \usepackage{amsmath} % Equations
    \usepackage{amssymb} % Equations
    \usepackage{textcomp} % defines textquotesingle
    % Hack from http://tex.stackexchange.com/a/47451/13684:
    \AtBeginDocument{%
        \def\PYZsq{\textquotesingle}% Upright quotes in Pygmentized code
    }
    \usepackage{upquote} % Upright quotes for verbatim code
    \usepackage{eurosym} % defines \euro

    \usepackage{iftex}
    \ifPDFTeX
        \usepackage[T1]{fontenc}
        \IfFileExists{alphabeta.sty}{
              \usepackage{alphabeta}
          }{
              \usepackage[mathletters]{ucs}
              \usepackage[utf8x]{inputenc}
          }
    \else
        \usepackage{fontspec}
        \usepackage{unicode-math}
    \fi

    \usepackage{fancyvrb} % verbatim replacement that allows latex
    \usepackage{grffile} % extends the file name processing of package graphics
                         % to support a larger range
    \makeatletter % fix for old versions of grffile with XeLaTeX
    \@ifpackagelater{grffile}{2019/11/01}
    {
      % Do nothing on new versions
    }
    {
      \def\Gread@@xetex#1{%
        \IfFileExists{"\Gin@base".bb}%
        {\Gread@eps{\Gin@base.bb}}%
        {\Gread@@xetex@aux#1}%
      }
    }
    \makeatother
    \usepackage[Export]{adjustbox} % Used to constrain images to a maximum size
    \adjustboxset{max size={0.9\linewidth}{0.9\paperheight}}

    % The hyperref package gives us a pdf with properly built
    % internal navigation ('pdf bookmarks' for the table of contents,
    % internal cross-reference links, web links for URLs, etc.)
    \usepackage{hyperref}
    % The default LaTeX title has an obnoxious amount of whitespace. By default,
    % titling removes some of it. It also provides customization options.
    \usepackage{titling}
    \usepackage{longtable} % longtable support required by pandoc >1.10
    \usepackage{booktabs}  % table support for pandoc > 1.12.2
    \usepackage{array}     % table support for pandoc >= 2.11.3
    \usepackage{calc}      % table minipage width calculation for pandoc >= 2.11.1
    \usepackage[inline]{enumitem} % IRkernel/repr support (it uses the enumerate* environment)
    \usepackage[normalem]{ulem} % ulem is needed to support strikethroughs (\sout)
                                % normalem makes italics be italics, not underlines
    \usepackage{soul}      % strikethrough (\st) support for pandoc >= 3.0.0
    \usepackage{mathrsfs}
    

    
    % Colors for the hyperref package
    \definecolor{urlcolor}{rgb}{0,.145,.698}
    \definecolor{linkcolor}{rgb}{.71,0.21,0.01}
    \definecolor{citecolor}{rgb}{.12,.54,.11}

    % ANSI colors
    \definecolor{ansi-black}{HTML}{3E424D}
    \definecolor{ansi-black-intense}{HTML}{282C36}
    \definecolor{ansi-red}{HTML}{E75C58}
    \definecolor{ansi-red-intense}{HTML}{B22B31}
    \definecolor{ansi-green}{HTML}{00A250}
    \definecolor{ansi-green-intense}{HTML}{007427}
    \definecolor{ansi-yellow}{HTML}{DDB62B}
    \definecolor{ansi-yellow-intense}{HTML}{B27D12}
    \definecolor{ansi-blue}{HTML}{208FFB}
    \definecolor{ansi-blue-intense}{HTML}{0065CA}
    \definecolor{ansi-magenta}{HTML}{D160C4}
    \definecolor{ansi-magenta-intense}{HTML}{A03196}
    \definecolor{ansi-cyan}{HTML}{60C6C8}
    \definecolor{ansi-cyan-intense}{HTML}{258F8F}
    \definecolor{ansi-white}{HTML}{C5C1B4}
    \definecolor{ansi-white-intense}{HTML}{A1A6B2}
    \definecolor{ansi-default-inverse-fg}{HTML}{FFFFFF}
    \definecolor{ansi-default-inverse-bg}{HTML}{000000}

    % common color for the border for error outputs.
    \definecolor{outerrorbackground}{HTML}{FFDFDF}

    % commands and environments needed by pandoc snippets
    % extracted from the output of `pandoc -s`
    \providecommand{\tightlist}{%
      \setlength{\itemsep}{0pt}\setlength{\parskip}{0pt}}
    \DefineVerbatimEnvironment{Highlighting}{Verbatim}{commandchars=\\\{\}}
    % Add ',fontsize=\small' for more characters per line
    \newenvironment{Shaded}{}{}
    \newcommand{\KeywordTok}[1]{\textcolor[rgb]{0.00,0.44,0.13}{\textbf{{#1}}}}
    \newcommand{\DataTypeTok}[1]{\textcolor[rgb]{0.56,0.13,0.00}{{#1}}}
    \newcommand{\DecValTok}[1]{\textcolor[rgb]{0.25,0.63,0.44}{{#1}}}
    \newcommand{\BaseNTok}[1]{\textcolor[rgb]{0.25,0.63,0.44}{{#1}}}
    \newcommand{\FloatTok}[1]{\textcolor[rgb]{0.25,0.63,0.44}{{#1}}}
    \newcommand{\CharTok}[1]{\textcolor[rgb]{0.25,0.44,0.63}{{#1}}}
    \newcommand{\StringTok}[1]{\textcolor[rgb]{0.25,0.44,0.63}{{#1}}}
    \newcommand{\CommentTok}[1]{\textcolor[rgb]{0.38,0.63,0.69}{\textit{{#1}}}}
    \newcommand{\OtherTok}[1]{\textcolor[rgb]{0.00,0.44,0.13}{{#1}}}
    \newcommand{\AlertTok}[1]{\textcolor[rgb]{1.00,0.00,0.00}{\textbf{{#1}}}}
    \newcommand{\FunctionTok}[1]{\textcolor[rgb]{0.02,0.16,0.49}{{#1}}}
    \newcommand{\RegionMarkerTok}[1]{{#1}}
    \newcommand{\ErrorTok}[1]{\textcolor[rgb]{1.00,0.00,0.00}{\textbf{{#1}}}}
    \newcommand{\NormalTok}[1]{{#1}}

    % Additional commands for more recent versions of Pandoc
    \newcommand{\ConstantTok}[1]{\textcolor[rgb]{0.53,0.00,0.00}{{#1}}}
    \newcommand{\SpecialCharTok}[1]{\textcolor[rgb]{0.25,0.44,0.63}{{#1}}}
    \newcommand{\VerbatimStringTok}[1]{\textcolor[rgb]{0.25,0.44,0.63}{{#1}}}
    \newcommand{\SpecialStringTok}[1]{\textcolor[rgb]{0.73,0.40,0.53}{{#1}}}
    \newcommand{\ImportTok}[1]{{#1}}
    \newcommand{\DocumentationTok}[1]{\textcolor[rgb]{0.73,0.13,0.13}{\textit{{#1}}}}
    \newcommand{\AnnotationTok}[1]{\textcolor[rgb]{0.38,0.63,0.69}{\textbf{\textit{{#1}}}}}
    \newcommand{\CommentVarTok}[1]{\textcolor[rgb]{0.38,0.63,0.69}{\textbf{\textit{{#1}}}}}
    \newcommand{\VariableTok}[1]{\textcolor[rgb]{0.10,0.09,0.49}{{#1}}}
    \newcommand{\ControlFlowTok}[1]{\textcolor[rgb]{0.00,0.44,0.13}{\textbf{{#1}}}}
    \newcommand{\OperatorTok}[1]{\textcolor[rgb]{0.40,0.40,0.40}{{#1}}}
    \newcommand{\BuiltInTok}[1]{{#1}}
    \newcommand{\ExtensionTok}[1]{{#1}}
    \newcommand{\PreprocessorTok}[1]{\textcolor[rgb]{0.74,0.48,0.00}{{#1}}}
    \newcommand{\AttributeTok}[1]{\textcolor[rgb]{0.49,0.56,0.16}{{#1}}}
    \newcommand{\InformationTok}[1]{\textcolor[rgb]{0.38,0.63,0.69}{\textbf{\textit{{#1}}}}}
    \newcommand{\WarningTok}[1]{\textcolor[rgb]{0.38,0.63,0.69}{\textbf{\textit{{#1}}}}}
    \makeatletter
    \newsavebox\pandoc@box
    \newcommand*\pandocbounded[1]{%
      \sbox\pandoc@box{#1}%
      % scaling factors for width and height
      \Gscale@div\@tempa\textheight{\dimexpr\ht\pandoc@box+\dp\pandoc@box\relax}%
      \Gscale@div\@tempb\linewidth{\wd\pandoc@box}%
      % select the smaller of both
      \ifdim\@tempb\p@<\@tempa\p@
        \let\@tempa\@tempb
      \fi
      % scaling accordingly (\@tempa < 1)
      \ifdim\@tempa\p@<\p@
        \scalebox{\@tempa}{\usebox\pandoc@box}%
      % scaling not needed, use as it is
      \else
        \usebox{\pandoc@box}%
      \fi
    }
    \makeatother

    % Define a nice break command that doesn't care if a line doesn't already
    % exist.
    \def\br{\hspace*{\fill} \\* }
    % Math Jax compatibility definitions
    \def\gt{>}
    \def\lt{<}
    \let\Oldtex\TeX
    \let\Oldlatex\LaTeX
    \renewcommand{\TeX}{\textrm{\Oldtex}}
    \renewcommand{\LaTeX}{\textrm{\Oldlatex}}
    % Document parameters
    % Document title
    \title{Group10 AmesHousing}
    
    
    
    
    
    
    
% Pygments definitions
\makeatletter
\def\PY@reset{\let\PY@it=\relax \let\PY@bf=\relax%
    \let\PY@ul=\relax \let\PY@tc=\relax%
    \let\PY@bc=\relax \let\PY@ff=\relax}
\def\PY@tok#1{\csname PY@tok@#1\endcsname}
\def\PY@toks#1+{\ifx\relax#1\empty\else%
    \PY@tok{#1}\expandafter\PY@toks\fi}
\def\PY@do#1{\PY@bc{\PY@tc{\PY@ul{%
    \PY@it{\PY@bf{\PY@ff{#1}}}}}}}
\def\PY#1#2{\PY@reset\PY@toks#1+\relax+\PY@do{#2}}

\@namedef{PY@tok@w}{\def\PY@tc##1{\textcolor[rgb]{0.73,0.73,0.73}{##1}}}
\@namedef{PY@tok@c}{\let\PY@it=\textit\def\PY@tc##1{\textcolor[rgb]{0.24,0.48,0.48}{##1}}}
\@namedef{PY@tok@cp}{\def\PY@tc##1{\textcolor[rgb]{0.61,0.40,0.00}{##1}}}
\@namedef{PY@tok@k}{\let\PY@bf=\textbf\def\PY@tc##1{\textcolor[rgb]{0.00,0.50,0.00}{##1}}}
\@namedef{PY@tok@kp}{\def\PY@tc##1{\textcolor[rgb]{0.00,0.50,0.00}{##1}}}
\@namedef{PY@tok@kt}{\def\PY@tc##1{\textcolor[rgb]{0.69,0.00,0.25}{##1}}}
\@namedef{PY@tok@o}{\def\PY@tc##1{\textcolor[rgb]{0.40,0.40,0.40}{##1}}}
\@namedef{PY@tok@ow}{\let\PY@bf=\textbf\def\PY@tc##1{\textcolor[rgb]{0.67,0.13,1.00}{##1}}}
\@namedef{PY@tok@nb}{\def\PY@tc##1{\textcolor[rgb]{0.00,0.50,0.00}{##1}}}
\@namedef{PY@tok@nf}{\def\PY@tc##1{\textcolor[rgb]{0.00,0.00,1.00}{##1}}}
\@namedef{PY@tok@nc}{\let\PY@bf=\textbf\def\PY@tc##1{\textcolor[rgb]{0.00,0.00,1.00}{##1}}}
\@namedef{PY@tok@nn}{\let\PY@bf=\textbf\def\PY@tc##1{\textcolor[rgb]{0.00,0.00,1.00}{##1}}}
\@namedef{PY@tok@ne}{\let\PY@bf=\textbf\def\PY@tc##1{\textcolor[rgb]{0.80,0.25,0.22}{##1}}}
\@namedef{PY@tok@nv}{\def\PY@tc##1{\textcolor[rgb]{0.10,0.09,0.49}{##1}}}
\@namedef{PY@tok@no}{\def\PY@tc##1{\textcolor[rgb]{0.53,0.00,0.00}{##1}}}
\@namedef{PY@tok@nl}{\def\PY@tc##1{\textcolor[rgb]{0.46,0.46,0.00}{##1}}}
\@namedef{PY@tok@ni}{\let\PY@bf=\textbf\def\PY@tc##1{\textcolor[rgb]{0.44,0.44,0.44}{##1}}}
\@namedef{PY@tok@na}{\def\PY@tc##1{\textcolor[rgb]{0.41,0.47,0.13}{##1}}}
\@namedef{PY@tok@nt}{\let\PY@bf=\textbf\def\PY@tc##1{\textcolor[rgb]{0.00,0.50,0.00}{##1}}}
\@namedef{PY@tok@nd}{\def\PY@tc##1{\textcolor[rgb]{0.67,0.13,1.00}{##1}}}
\@namedef{PY@tok@s}{\def\PY@tc##1{\textcolor[rgb]{0.73,0.13,0.13}{##1}}}
\@namedef{PY@tok@sd}{\let\PY@it=\textit\def\PY@tc##1{\textcolor[rgb]{0.73,0.13,0.13}{##1}}}
\@namedef{PY@tok@si}{\let\PY@bf=\textbf\def\PY@tc##1{\textcolor[rgb]{0.64,0.35,0.47}{##1}}}
\@namedef{PY@tok@se}{\let\PY@bf=\textbf\def\PY@tc##1{\textcolor[rgb]{0.67,0.36,0.12}{##1}}}
\@namedef{PY@tok@sr}{\def\PY@tc##1{\textcolor[rgb]{0.64,0.35,0.47}{##1}}}
\@namedef{PY@tok@ss}{\def\PY@tc##1{\textcolor[rgb]{0.10,0.09,0.49}{##1}}}
\@namedef{PY@tok@sx}{\def\PY@tc##1{\textcolor[rgb]{0.00,0.50,0.00}{##1}}}
\@namedef{PY@tok@m}{\def\PY@tc##1{\textcolor[rgb]{0.40,0.40,0.40}{##1}}}
\@namedef{PY@tok@gh}{\let\PY@bf=\textbf\def\PY@tc##1{\textcolor[rgb]{0.00,0.00,0.50}{##1}}}
\@namedef{PY@tok@gu}{\let\PY@bf=\textbf\def\PY@tc##1{\textcolor[rgb]{0.50,0.00,0.50}{##1}}}
\@namedef{PY@tok@gd}{\def\PY@tc##1{\textcolor[rgb]{0.63,0.00,0.00}{##1}}}
\@namedef{PY@tok@gi}{\def\PY@tc##1{\textcolor[rgb]{0.00,0.52,0.00}{##1}}}
\@namedef{PY@tok@gr}{\def\PY@tc##1{\textcolor[rgb]{0.89,0.00,0.00}{##1}}}
\@namedef{PY@tok@ge}{\let\PY@it=\textit}
\@namedef{PY@tok@gs}{\let\PY@bf=\textbf}
\@namedef{PY@tok@ges}{\let\PY@bf=\textbf\let\PY@it=\textit}
\@namedef{PY@tok@gp}{\let\PY@bf=\textbf\def\PY@tc##1{\textcolor[rgb]{0.00,0.00,0.50}{##1}}}
\@namedef{PY@tok@go}{\def\PY@tc##1{\textcolor[rgb]{0.44,0.44,0.44}{##1}}}
\@namedef{PY@tok@gt}{\def\PY@tc##1{\textcolor[rgb]{0.00,0.27,0.87}{##1}}}
\@namedef{PY@tok@err}{\def\PY@bc##1{{\setlength{\fboxsep}{\string -\fboxrule}\fcolorbox[rgb]{1.00,0.00,0.00}{1,1,1}{\strut ##1}}}}
\@namedef{PY@tok@kc}{\let\PY@bf=\textbf\def\PY@tc##1{\textcolor[rgb]{0.00,0.50,0.00}{##1}}}
\@namedef{PY@tok@kd}{\let\PY@bf=\textbf\def\PY@tc##1{\textcolor[rgb]{0.00,0.50,0.00}{##1}}}
\@namedef{PY@tok@kn}{\let\PY@bf=\textbf\def\PY@tc##1{\textcolor[rgb]{0.00,0.50,0.00}{##1}}}
\@namedef{PY@tok@kr}{\let\PY@bf=\textbf\def\PY@tc##1{\textcolor[rgb]{0.00,0.50,0.00}{##1}}}
\@namedef{PY@tok@bp}{\def\PY@tc##1{\textcolor[rgb]{0.00,0.50,0.00}{##1}}}
\@namedef{PY@tok@fm}{\def\PY@tc##1{\textcolor[rgb]{0.00,0.00,1.00}{##1}}}
\@namedef{PY@tok@vc}{\def\PY@tc##1{\textcolor[rgb]{0.10,0.09,0.49}{##1}}}
\@namedef{PY@tok@vg}{\def\PY@tc##1{\textcolor[rgb]{0.10,0.09,0.49}{##1}}}
\@namedef{PY@tok@vi}{\def\PY@tc##1{\textcolor[rgb]{0.10,0.09,0.49}{##1}}}
\@namedef{PY@tok@vm}{\def\PY@tc##1{\textcolor[rgb]{0.10,0.09,0.49}{##1}}}
\@namedef{PY@tok@sa}{\def\PY@tc##1{\textcolor[rgb]{0.73,0.13,0.13}{##1}}}
\@namedef{PY@tok@sb}{\def\PY@tc##1{\textcolor[rgb]{0.73,0.13,0.13}{##1}}}
\@namedef{PY@tok@sc}{\def\PY@tc##1{\textcolor[rgb]{0.73,0.13,0.13}{##1}}}
\@namedef{PY@tok@dl}{\def\PY@tc##1{\textcolor[rgb]{0.73,0.13,0.13}{##1}}}
\@namedef{PY@tok@s2}{\def\PY@tc##1{\textcolor[rgb]{0.73,0.13,0.13}{##1}}}
\@namedef{PY@tok@sh}{\def\PY@tc##1{\textcolor[rgb]{0.73,0.13,0.13}{##1}}}
\@namedef{PY@tok@s1}{\def\PY@tc##1{\textcolor[rgb]{0.73,0.13,0.13}{##1}}}
\@namedef{PY@tok@mb}{\def\PY@tc##1{\textcolor[rgb]{0.40,0.40,0.40}{##1}}}
\@namedef{PY@tok@mf}{\def\PY@tc##1{\textcolor[rgb]{0.40,0.40,0.40}{##1}}}
\@namedef{PY@tok@mh}{\def\PY@tc##1{\textcolor[rgb]{0.40,0.40,0.40}{##1}}}
\@namedef{PY@tok@mi}{\def\PY@tc##1{\textcolor[rgb]{0.40,0.40,0.40}{##1}}}
\@namedef{PY@tok@il}{\def\PY@tc##1{\textcolor[rgb]{0.40,0.40,0.40}{##1}}}
\@namedef{PY@tok@mo}{\def\PY@tc##1{\textcolor[rgb]{0.40,0.40,0.40}{##1}}}
\@namedef{PY@tok@ch}{\let\PY@it=\textit\def\PY@tc##1{\textcolor[rgb]{0.24,0.48,0.48}{##1}}}
\@namedef{PY@tok@cm}{\let\PY@it=\textit\def\PY@tc##1{\textcolor[rgb]{0.24,0.48,0.48}{##1}}}
\@namedef{PY@tok@cpf}{\let\PY@it=\textit\def\PY@tc##1{\textcolor[rgb]{0.24,0.48,0.48}{##1}}}
\@namedef{PY@tok@c1}{\let\PY@it=\textit\def\PY@tc##1{\textcolor[rgb]{0.24,0.48,0.48}{##1}}}
\@namedef{PY@tok@cs}{\let\PY@it=\textit\def\PY@tc##1{\textcolor[rgb]{0.24,0.48,0.48}{##1}}}

\def\PYZbs{\char`\\}
\def\PYZus{\char`\_}
\def\PYZob{\char`\{}
\def\PYZcb{\char`\}}
\def\PYZca{\char`\^}
\def\PYZam{\char`\&}
\def\PYZlt{\char`\<}
\def\PYZgt{\char`\>}
\def\PYZsh{\char`\#}
\def\PYZpc{\char`\%}
\def\PYZdl{\char`\$}
\def\PYZhy{\char`\-}
\def\PYZsq{\char`\'}
\def\PYZdq{\char`\"}
\def\PYZti{\char`\~}
% for compatibility with earlier versions
\def\PYZat{@}
\def\PYZlb{[}
\def\PYZrb{]}
\makeatother


    % For linebreaks inside Verbatim environment from package fancyvrb.
    \makeatletter
        \newbox\Wrappedcontinuationbox
        \newbox\Wrappedvisiblespacebox
        \newcommand*\Wrappedvisiblespace {\textcolor{red}{\textvisiblespace}}
        \newcommand*\Wrappedcontinuationsymbol {\textcolor{red}{\llap{\tiny$\m@th\hookrightarrow$}}}
        \newcommand*\Wrappedcontinuationindent {3ex }
        \newcommand*\Wrappedafterbreak {\kern\Wrappedcontinuationindent\copy\Wrappedcontinuationbox}
        % Take advantage of the already applied Pygments mark-up to insert
        % potential linebreaks for TeX processing.
        %        {, <, #, %, $, ' and ": go to next line.
        %        _, }, ^, &, >, - and ~: stay at end of broken line.
        % Use of \textquotesingle for straight quote.
        \newcommand*\Wrappedbreaksatspecials {%
            \def\PYGZus{\discretionary{\char`\_}{\Wrappedafterbreak}{\char`\_}}%
            \def\PYGZob{\discretionary{}{\Wrappedafterbreak\char`\{}{\char`\{}}%
            \def\PYGZcb{\discretionary{\char`\}}{\Wrappedafterbreak}{\char`\}}}%
            \def\PYGZca{\discretionary{\char`\^}{\Wrappedafterbreak}{\char`\^}}%
            \def\PYGZam{\discretionary{\char`\&}{\Wrappedafterbreak}{\char`\&}}%
            \def\PYGZlt{\discretionary{}{\Wrappedafterbreak\char`\<}{\char`\<}}%
            \def\PYGZgt{\discretionary{\char`\>}{\Wrappedafterbreak}{\char`\>}}%
            \def\PYGZsh{\discretionary{}{\Wrappedafterbreak\char`\#}{\char`\#}}%
            \def\PYGZpc{\discretionary{}{\Wrappedafterbreak\char`\%}{\char`\%}}%
            \def\PYGZdl{\discretionary{}{\Wrappedafterbreak\char`\$}{\char`\$}}%
            \def\PYGZhy{\discretionary{\char`\-}{\Wrappedafterbreak}{\char`\-}}%
            \def\PYGZsq{\discretionary{}{\Wrappedafterbreak\textquotesingle}{\textquotesingle}}%
            \def\PYGZdq{\discretionary{}{\Wrappedafterbreak\char`\"}{\char`\"}}%
            \def\PYGZti{\discretionary{\char`\~}{\Wrappedafterbreak}{\char`\~}}%
        }
        % Some characters . , ; ? ! / are not pygmentized.
        % This macro makes them "active" and they will insert potential linebreaks
        \newcommand*\Wrappedbreaksatpunct {%
            \lccode`\~`\.\lowercase{\def~}{\discretionary{\hbox{\char`\.}}{\Wrappedafterbreak}{\hbox{\char`\.}}}%
            \lccode`\~`\,\lowercase{\def~}{\discretionary{\hbox{\char`\,}}{\Wrappedafterbreak}{\hbox{\char`\,}}}%
            \lccode`\~`\;\lowercase{\def~}{\discretionary{\hbox{\char`\;}}{\Wrappedafterbreak}{\hbox{\char`\;}}}%
            \lccode`\~`\:\lowercase{\def~}{\discretionary{\hbox{\char`\:}}{\Wrappedafterbreak}{\hbox{\char`\:}}}%
            \lccode`\~`\?\lowercase{\def~}{\discretionary{\hbox{\char`\?}}{\Wrappedafterbreak}{\hbox{\char`\?}}}%
            \lccode`\~`\!\lowercase{\def~}{\discretionary{\hbox{\char`\!}}{\Wrappedafterbreak}{\hbox{\char`\!}}}%
            \lccode`\~`\/\lowercase{\def~}{\discretionary{\hbox{\char`\/}}{\Wrappedafterbreak}{\hbox{\char`\/}}}%
            \catcode`\.\active
            \catcode`\,\active
            \catcode`\;\active
            \catcode`\:\active
            \catcode`\?\active
            \catcode`\!\active
            \catcode`\/\active
            \lccode`\~`\~
        }
    \makeatother

    \let\OriginalVerbatim=\Verbatim
    \makeatletter
    \renewcommand{\Verbatim}[1][1]{%
        %\parskip\z@skip
        \sbox\Wrappedcontinuationbox {\Wrappedcontinuationsymbol}%
        \sbox\Wrappedvisiblespacebox {\FV@SetupFont\Wrappedvisiblespace}%
        \def\FancyVerbFormatLine ##1{\hsize\linewidth
            \vtop{\raggedright\hyphenpenalty\z@\exhyphenpenalty\z@
                \doublehyphendemerits\z@\finalhyphendemerits\z@
                \strut ##1\strut}%
        }%
        % If the linebreak is at a space, the latter will be displayed as visible
        % space at end of first line, and a continuation symbol starts next line.
        % Stretch/shrink are however usually zero for typewriter font.
        \def\FV@Space {%
            \nobreak\hskip\z@ plus\fontdimen3\font minus\fontdimen4\font
            \discretionary{\copy\Wrappedvisiblespacebox}{\Wrappedafterbreak}
            {\kern\fontdimen2\font}%
        }%

        % Allow breaks at special characters using \PYG... macros.
        \Wrappedbreaksatspecials
        % Breaks at punctuation characters . , ; ? ! and / need catcode=\active
        \OriginalVerbatim[#1,codes*=\Wrappedbreaksatpunct]%
    }
    \makeatother

    % Exact colors from NB
    \definecolor{incolor}{HTML}{303F9F}
    \definecolor{outcolor}{HTML}{D84315}
    \definecolor{cellborder}{HTML}{CFCFCF}
    \definecolor{cellbackground}{HTML}{F7F7F7}

    % prompt
    \makeatletter
    \newcommand{\boxspacing}{\kern\kvtcb@left@rule\kern\kvtcb@boxsep}
    \makeatother
    \newcommand{\prompt}[4]{
        {\ttfamily\llap{{\color{#2}[#3]:\hspace{3pt}#4}}\vspace{-\baselineskip}}
    }
    

    
    % Prevent overflowing lines due to hard-to-break entities
    \sloppy
    % Setup hyperref package
    \hypersetup{
      breaklinks=true,  % so long urls are correctly broken across lines
      colorlinks=true,
      urlcolor=urlcolor,
      linkcolor=linkcolor,
      citecolor=citecolor,
      }
    % Slightly bigger margins than the latex defaults
    
    \geometry{verbose,tmargin=1in,bmargin=1in,lmargin=1in,rmargin=1in}
    
    

\begin{document}
    
    \maketitle
    
    

    
    \hypertarget{research-hypotheses-supported-by-the-literature}{%
\section{Research hypotheses supported by the
literature}\label{research-hypotheses-supported-by-the-literature}}

The objective of this work is to study which structural, qualitative,
and locational characteristics of a residential property significantly
affect its market price. \cite{glasserman2004}

The \emph{Ames Housing Dataset} provides detailed information on
dwellings sold in Ames (Iowa) and is an updated version of the Boston
Housing Dataset, already widely used in the housing econometrics
literature. With its 2930, the \emph{Ames Housing Dataset} offers a
larger sample size and a more comprehensive set of features and it is
better suited for modern statistical analysis.

Ames dataset includes physical size, construction details, quality
ratings, and neighborhood indicators, all variables on which we
formulate a series of hypotheses grounded in empirical findings from the
real-estate and housing-econometrics literature.

    \hypertarget{fundamentals-variables-and-assumptions}{%
\subsection{Fundamentals variables and
assumptions}\label{fundamentals-variables-and-assumptions}}

We found that about 80\% of the variation in residential sales price can
be explained by simply taking into consideration the neighborhood and
total square footage (\texttt{Total\ Bsmt\ SF} + \texttt{Gr\ Liv\ Area})
of the dwelling (e.g., De Cock 2011).

According to our interpretation of the literature published (e.g., Ju et
Al 2025; Zietz et Sirmans 2008) and our empirical reasoning, we consider
the variables that significantly affect the sale price of a house are
the following.

    \hypertarget{the-overall-quality-of-the-house-has-a-strong-positive-impact-on-sale-price}{%
\subsubsection{The overall quality of the house has a strong positive
impact on sale
price}\label{the-overall-quality-of-the-house-has-a-strong-positive-impact-on-sale-price}}

Several studies show that global quality assessments summarise multiple
latent characteristics (materials, workmanship, design), making them
among the most informative predictors of house value.

Abdulhafedh (2022) identifies \texttt{OverallQual} as one of the most
influential variables in explaining sale price in a multiple regression
framework.

Similar evidence is reported in Yu (2022) and Han (2023), where overall
quality consistently appears as the strongest determinant in both OLS
and regularized regression models.

    \hypertarget{the-above-ground-living-area-grlivarea-is-positively-associated-with-the-price-of-a-house}{%
\subsubsection{\texorpdfstring{The above-ground living area
(\texttt{GrLivArea}) is positively associated with the price of a
house}{The above-ground living area (GrLivArea) is positively associated with the price of a house}}\label{the-above-ground-living-area-grlivarea-is-positively-associated-with-the-price-of-a-house}}

The hedonic pricing literature traditionally recognises physical size as
a primary contributor to housing value.

\begin{quote}
The \textbf{hedonic pricing model} is an economic model that explains
the price of a good as the sum (or combination) of the values of its
attributes. In other words, a complex good is ``decomposed'' into its
components, and the total price reflects the contribution of each of
them.

The concept was introduced by Lancaster in 1966 in consumer theory and
was first applied to housing prices by Rosen in 1974.
\end{quote}

Abdulhafedh (2022) highlights that the main livable area is among the
variables with the highest explanatory power. Studies on the Ames
dataset (e.g., Ye 2024; Han 2023) similarly identify \texttt{GrLivArea}
as one of the strongest continuous predictors.

    \hypertarget{the-total-basement-area-totalbsmtsf-also-increases-house-price-independently-from-the-above-ground-area}{%
\subsubsection{\texorpdfstring{The total basement area
(\texttt{TotalBsmtSF}) also increases house price, independently from
the above-ground
area}{The total basement area (TotalBsmtSF) also increases house price, independently from the above-ground area}}\label{the-total-basement-area-totalbsmtsf-also-increases-house-price-independently-from-the-above-ground-area}}

Basements provide additional functional space and generally correlate
with larger, higher-value homes.

Multiple analyses of the Ames dataset (Han 2023; Yu 2022) show that
basement size maintains statistical significance even when controlling
for other structural variables. Abdulhafedh (2022) also finds that
basement-related features contribute substantially to the model fit.

    \hypertarget{newer-or-recently-renovated-houses-yearbuilt-yearremodadd-tend-to-have-higher-sale-prices}{%
\subsubsection{\texorpdfstring{Newer or recently renovated houses
(\texttt{YearBuilt}, \texttt{YearRemodAdd}) tend to have higher sale
prices}{Newer or recently renovated houses (YearBuilt, YearRemodAdd) tend to have higher sale prices}}\label{newer-or-recently-renovated-houses-yearbuilt-yearremodadd-tend-to-have-higher-sale-prices}}

The literature on housing depreciation demonstrates that structural
aging reduces property value unless offset by renovations.

According to Abdulhafedh (2022), both the construction year and the
remodeling year play an important role in predicting sale prices.

Other studies on the Ames dataset reinforce this conclusion, noting that
newer homes or homes with extensive remodeling command a price premium.

    \hypertarget{higher-quality-kitchens-and-exterior-materials-positively-affect-sale-price}{%
\subsubsection{Higher-quality kitchens and exterior materials positively
affect sale
price}\label{higher-quality-kitchens-and-exterior-materials-positively-affect-sale-price}}

Studies in real-estate economics show that buyers are particularly
sensitive to the quality of kitchens, bathrooms, and exterior finishes,
as these elements influence both functionality and aesthetic appeal.

\texttt{KitchenQual} and \texttt{ExterQual} are repeatedly found to be
statistically significant predictors in analyses using the Ames dataset
(Ye 2024; Abdulhafedh 2022). Both variables capture qualitative
assessments not reflected solely by house size.

    \hypertarget{neighborhood-characteristics-significantly-affect-sale-price-even-after-controlling-for-structural-attributes.}{%
\subsubsection{Neighborhood characteristics significantly affect sale
price, even after controlling for structural
attributes.}\label{neighborhood-characteristics-significantly-affect-sale-price-even-after-controlling-for-structural-attributes.}}

Location remains one of the most important determinants of housing
prices in hedonic models.

The Ames dataset includes a categorical variable (\texttt{Neighborhood})
that encodes proximity to schools, income areas, and local amenities.

Previous studies (e.g., De Cock 2011; Ye 2024) demonstrate that location
dummies remain significant even in fully specified regression models.

    \hypertarget{variables-we-remove}{%
\subsection{Variables we remove}\label{variables-we-remove}}

First, we remove the variables \texttt{Pool\ QC}, \texttt{Alley},
\texttt{Fence} and \texttt{Misc\ Feature} since they contain too many
missing values (\texttt{NaN}). The remaining variables would be
available and ready for analysis but, as we have shown, including
irrelevant variables increases the standard errors, which leads to less
powerful significance tests and wider confidence intervals, and
therefore to less precise statistical inference.

For this reason we have not considered many of the 82 explanatory
variables, mainly because many of them refer to the same part of the
house (such as \texttt{Bsmt\ Exposure}, \texttt{BsmtFin\ Type\ 1},
\texttt{BsmtFin\ Type\ 2} for the description of the basement), and
therefore may be highly correlated and they generate high standard
errors, also because we belive that some of them have a negligeable
impact on sale prices. In order to obtain a more parsimonious model,
with an higher \(R^2\) adjusted and lower \(AIC\) and \(BIC\) we have
decided to exclude these variables from the analysis.

An examlple of such variables are:

\begin{itemize}
\tightlist
\item
  \texttt{PID} is just an identification number wich does not provide
  any information about the house price
\item
  \texttt{Lot\ Front} is the length of street connected to the property,
  which is not relevant for the price (on one hand a larger front may be
  better, on the other hand it may mean more noise and traffic)
\item
  \texttt{Lot\ Shape} and \texttt{Land\ Contour} are less relevant than
  other locational attributes like neighborhood
\item
  \texttt{Fireplaces} is the number of fireplaces, which is an outdated
  feature that does not affect the price significantly
\item
  \texttt{Fireplace\ Qu} is the quality of the fireplaces, which is of
  secondary importance compared to other quality features like kitchen
  and exterior quality
\item
  \texttt{Roof\ Style} and \texttt{Roof\ Matl} are of secondary
  importance compared to other features such as quality and size
\item
  \texttt{Condition\ 1} indicates the proximity to various conditions
  (e.g., arterial roads, railroads), but \texttt{Neighborhood} already
  captures location effects more comprehensively plus there is still the
  noise factor
\end{itemize}

    \hypertarget{description-of-the-dataset}{%
\section{Description of the Dataset}\label{description-of-the-dataset}}

The analysis is based on the \emph{Ames Housing Dataset}, a well-known
real-estate dataset originally compiled by the Assessor's Office of
Ames, Iowa. It contains detailed information on residential properties
sold between 2006 and 2010.

In its full version, the dataset includes 82 explanatory variables
describing structural, qualitative, locational, and functional
characteristics of each house.

For the purpose of this assignment, we selected a subset of the
variables most commonly used in hedonic price models and supported by
the literature.

    \hypertarget{general-structure-of-the-dataset}{%
\subsection{General structure of the
dataset}\label{general-structure-of-the-dataset}}

Each row of the dataset corresponds to a single residential property,
while columns represent the attributes listed above.

Numerical variables are measured either in square feet or in calendar
years, whereas qualitative assessments use an ordinal scale. The dataset
does not contain missing values for the variables selected for the
analysis, and therefore no imputation was required in this step.

The dependent variable of the regression is \texttt{SalePrice},
expressed in US dollars. and the selected explanatory variables fall
into the four following categories.

    \hypertarget{size-and-structural-characteristics}{%
\subsubsection{Size and structural
characteristics}\label{size-and-structural-characteristics}}

These variables capture the physical dimensions of the dwelling:

\begin{itemize}
\tightlist
\item
  \texttt{Gr\ Liv\ Area}: above-ground living area (in square feet)
\item
  \texttt{1st\ Flr\ SF}: first-floor area
\item
  \texttt{Total\ Bsmt\ SF}: total basement area
\item
  \texttt{Lot\ Area}: size of the lot
\item
  \texttt{Full\ Bath}: number of full bathrooms
\item
  \texttt{Garage\ Area}: size of the garage
\item
  \texttt{Garage\ Cars}: garage capacity (number of cars)
\item
  \texttt{Garage\ Yr\ Blt}: year the garage was built
\item
  \texttt{Year\ Built}: construction year of the house
\item
  \texttt{Year\ Remod/Add}: year of the most recent remodeling
\item
  \texttt{Utilities}: type of utilities available
\end{itemize}

These attributes quantify the amount of usable space and the structural
age of the property.

    \hypertarget{quality-assessments}{%
\subsubsection{Quality assessments}\label{quality-assessments}}

The dataset includes several ordinal ratings assigned by professional
assessors:

\begin{itemize}
\tightlist
\item
  \texttt{Overall\ Qual}: overall material and finish quality
\item
  \texttt{Kitchen\ Qual}: quality of kitchen materials
\item
  \texttt{ExterQual}: assessment of exterior materials and workmanship
\item
  \texttt{BsmtQual}: quality of basement materials
\end{itemize}

These variables summarize qualitative features that are not captured by
size alone.

    \hypertarget{locational-and-timing-information}{%
\subsubsection{Locational and timing
information}\label{locational-and-timing-information}}

\begin{itemize}
\tightlist
\item
  \texttt{Neighborhood}: categorical variable identifying the physical
  location within the city of Ames
\item
  \texttt{MS\ Zoning}: categorical variable indicating the general
  zoning classification such as low‑density residential, medium‑density
  residential, or commercial
\item
  \texttt{Year\ Sold}: year the property was sold
\end{itemize}

This variable incorporates locational amenities and neighbourhood-level
characteristics that influence market value.

    \hypertarget{model-specification-and-estimation}{%
\section{Model Specification and
Estimation}\label{model-specification-and-estimation}}

The aim of this section is to construct a multiple linear regression
model in order to explain the variation in the sale price of residential
properties.

The dependent variable is \texttt{SalePrice}, while the set of
regressors includes the structural, qualitative, and locational
characteristics identified in the previous section.

    \hypertarget{model-specification}{%
\subsection{Model Specification}\label{model-specification}}

The regression model is specified as follows:

\[
\begin{aligned}
\mathrm{SalePrice}_i &= 
\beta_0
+ \beta_1\,\mathrm{Gr Liv Area}_i
+ \beta_2\,\mathrm{1st Flr SF}_i
+ \beta_3\,\mathrm{Total Bsmt SF}_i
+ \beta_4\,\mathrm{Lot Area}_i\\
&\quad + \beta_5\,\mathrm{Full Bath}_i
+ \beta_6\,\mathrm{Garage Area}_i
+ \beta_7\,\mathrm{Garage Cars}_i
+ \beta_8\,\mathrm{Garage Yr Blt}_i\\
&\quad + \beta_9\,\mathrm{Year Built}_i
+ \beta_{10}\,\mathrm{Year Remod/Add}_i
+ \beta_{11}\,\mathrm{Utilities}_i
+ \beta_{12}\,\mathrm{Overall Qual}_i\\
&\quad + \beta_{13}\,\mathrm{Kitchen Qual}_i
+ \beta_{14}\,\mathrm{Exter Qual}_i
+ \beta_{15}\,\mathrm{Bsmt Qual}_i
+ \beta_{15}\,\mathrm{Neighborhood}_i\\
&\quad + \beta_{16}\,\mathrm{MS Zoning}_i
+ \beta_{16}\,\mathrm{Year Sold}_i
+ \varepsilon_i
\end{aligned}
\]

where:

\begin{itemize}
\tightlist
\item
  \(\beta_0\) is the intercept
\item
  \(\beta\)'s denote the coefficients associated with each regressor
\item
  \(\varepsilon_i\) is the error term capturing unobserved factors
\item
  the \texttt{Neighborhood} variable is treated as a categorical factor
  and encoded through dummy variables
\end{itemize}

    \hypertarget{model-assumptions}{%
\subsection{Model Assumptions}\label{model-assumptions}}

Our multiple linear regression model is estimated under the following
assumptions:

\begin{itemize}
\item
  linearity of the \emph{conditional mean} \[
  \mathbb{E}[\varepsilon_i \mid X_i] = 0
  \ \Longleftrightarrow\ 
  \mathbb{E}[Y_i \mid X_i]
  = \beta_0 + \beta_1 X_{i1} + \cdots + \beta_p X_{ip}
  \] This assumption implies that all systematic variation in the
  dependent variable is captured by the regressors.
\item
  \emph{independent and identically distributed} observations \[
  (Y_i, X_i) \;\perp\!\!\!\perp\; (Y_j, X_j)
  \quad \text{for } i \neq j
  \] and all observations are drawn from the same population
  distribution. This corresponds to the standard \emph{i.i.d.} sampling
  framework.
\item
  finite \emph{fourth moments} \[
  \mathbb{E}[X_{ik}^4] < \infty
  \qquad\text{and}\qquad
  \mathbb{E}[\varepsilon_i^4] < \infty
  \] This condition ensures the applicability of asymptotic results and
  prevents extreme observations from dominating the estimation.
\end{itemize}

    \hypertarget{python-code}{%
\subsection{Python code}\label{python-code}}

Here we list the full code used for the analysis with all comments and
explanations.

    \begin{tcolorbox}[breakable, size=fbox, boxrule=1pt, pad at break*=1mm,colback=cellbackground, colframe=cellborder]
\prompt{In}{incolor}{1}{\boxspacing}
\begin{Verbatim}[commandchars=\\\{\}]
\PY{k+kn}{import}\PY{+w}{ }\PY{n+nn}{pandas}\PY{+w}{ }\PY{k}{as}\PY{+w}{ }\PY{n+nn}{pd}
\PY{k+kn}{import}\PY{+w}{ }\PY{n+nn}{statsmodels}\PY{n+nn}{.}\PY{n+nn}{api}\PY{+w}{ }\PY{k}{as}\PY{+w}{ }\PY{n+nn}{sm}
\PY{k+kn}{from}\PY{+w}{ }\PY{n+nn}{IPython}\PY{n+nn}{.}\PY{n+nn}{display}\PY{+w}{ }\PY{k+kn}{import} \PY{n}{display}

\PY{c+c1}{\PYZsh{} Read AmesHousing.csv into the DataFrame df:}
\PY{c+c1}{\PYZsh{} each row represents a house}
\PY{c+c1}{\PYZsh{} each column represents a variable}
\PY{n}{df} \PY{o}{=} \PY{n}{pd}\PY{o}{.}\PY{n}{read\PYZus{}csv}\PY{p}{(}\PY{l+s+s2}{\PYZdq{}}\PY{l+s+s2}{AmesHousing.csv}\PY{l+s+s2}{\PYZdq{}}\PY{p}{)}
\end{Verbatim}
\end{tcolorbox}

    \begin{tcolorbox}[breakable, size=fbox, boxrule=1pt, pad at break*=1mm,colback=cellbackground, colframe=cellborder]
\prompt{In}{incolor}{ }{\boxspacing}
\begin{Verbatim}[commandchars=\\\{\}]
\PY{c+c1}{\PYZsh{} SalePricee is the target variable y}
\PY{c+c1}{\PYZsh{} (the dependent variable)}
\PY{n}{y} \PY{o}{=} \PY{n}{df}\PY{p}{[}\PY{l+s+s2}{\PYZdq{}}\PY{l+s+s2}{SalePrice}\PY{l+s+s2}{\PYZdq{}}\PY{p}{]}

\PY{c+c1}{\PYZsh{} regressors is a list of column names that I want to use as regressors in the model,}
\PY{c+c1}{\PYZsh{} chosen from the 82 variables available in the dataset}
\PY{n}{regressors} \PY{o}{=} \PY{p}{[}
    \PY{l+s+s2}{\PYZdq{}}\PY{l+s+s2}{Gr Liv Area}\PY{l+s+s2}{\PYZdq{}}\PY{p}{,}
    \PY{l+s+s2}{\PYZdq{}}\PY{l+s+s2}{1st Flr SF}\PY{l+s+s2}{\PYZdq{}}\PY{p}{,}
    \PY{l+s+s2}{\PYZdq{}}\PY{l+s+s2}{Total Bsmt SF}\PY{l+s+s2}{\PYZdq{}}\PY{p}{,}
    \PY{l+s+s2}{\PYZdq{}}\PY{l+s+s2}{Lot Area}\PY{l+s+s2}{\PYZdq{}}\PY{p}{,}
    \PY{l+s+s2}{\PYZdq{}}\PY{l+s+s2}{Full Bath}\PY{l+s+s2}{\PYZdq{}}\PY{p}{,}
    \PY{l+s+s2}{\PYZdq{}}\PY{l+s+s2}{Garage Area}\PY{l+s+s2}{\PYZdq{}}\PY{p}{,}
    \PY{l+s+s2}{\PYZdq{}}\PY{l+s+s2}{Garage Cars}\PY{l+s+s2}{\PYZdq{}}\PY{p}{,}
    \PY{l+s+s2}{\PYZdq{}}\PY{l+s+s2}{Garage Yr Blt}\PY{l+s+s2}{\PYZdq{}}\PY{p}{,}
    \PY{l+s+s2}{\PYZdq{}}\PY{l+s+s2}{Year Built}\PY{l+s+s2}{\PYZdq{}}\PY{p}{,}
    \PY{l+s+s2}{\PYZdq{}}\PY{l+s+s2}{Year Remod/Add}\PY{l+s+s2}{\PYZdq{}}\PY{p}{,}
    \PY{l+s+s2}{\PYZdq{}}\PY{l+s+s2}{Overall Qual}\PY{l+s+s2}{\PYZdq{}}\PY{p}{,}
    \PY{l+s+s2}{\PYZdq{}}\PY{l+s+s2}{Kitchen Qual}\PY{l+s+s2}{\PYZdq{}}\PY{p}{,}
    \PY{l+s+s2}{\PYZdq{}}\PY{l+s+s2}{Exter Qual}\PY{l+s+s2}{\PYZdq{}}\PY{p}{,}
    \PY{l+s+s2}{\PYZdq{}}\PY{l+s+s2}{Bsmt Qual}\PY{l+s+s2}{\PYZdq{}}\PY{p}{,}
    \PY{l+s+s2}{\PYZdq{}}\PY{l+s+s2}{Neighborhood}\PY{l+s+s2}{\PYZdq{}}\PY{p}{,}
    \PY{l+s+s2}{\PYZdq{}}\PY{l+s+s2}{MS Zoning}\PY{l+s+s2}{\PYZdq{}}\PY{p}{,}
    \PY{l+s+s2}{\PYZdq{}}\PY{l+s+s2}{Utilities}\PY{l+s+s2}{\PYZdq{}}\PY{p}{,}
\PY{p}{]}
\end{Verbatim}
\end{tcolorbox}

    \begin{tcolorbox}[breakable, size=fbox, boxrule=1pt, pad at break*=1mm,colback=cellbackground, colframe=cellborder]
\prompt{In}{incolor}{3}{\boxspacing}
\begin{Verbatim}[commandchars=\\\{\}]
\PY{c+c1}{\PYZsh{} X is a new DataFrame that contains only the columns listed in regressors}
\PY{n}{X} \PY{o}{=} \PY{n}{df}\PY{p}{[}\PY{n}{regressors}\PY{p}{]}

\PY{c+c1}{\PYZsh{} Create dummy variables for the categorical variables:}
\PY{c+c1}{\PYZsh{} get\PYZus{}dummies converts categorical variables (strings) into a set of binary dummy columns}
\PY{c+c1}{\PYZsh{} \PYZdq{}Neighborhood\PYZdq{} \PYZhy{}\PYZgt{} [\PYZdq{}Neighborhood\PYZus{}Blueste\PYZdq{}, \PYZdq{}Neighborhood\PYZus{}BrDale\PYZdq{}, ...]}
\PY{c+c1}{\PYZsh{} \PYZdq{}House Style\PYZdq{}  \PYZhy{}\PYZgt{} [\PYZdq{}House Style\PYZus{}1.5Unf\PYZdq{}, \PYZdq{}House Style\PYZus{}1Story\PYZdq{}, ...]}
\PY{c+c1}{\PYZsh{} \PYZdq{}Kitchen Qual\PYZdq{} \PYZhy{}\PYZgt{} [\PYZdq{}Kitchen Qual\PYZus{}Fa\PYZdq{}, \PYZdq{}Kitchen Qual\PYZus{}Gd\PYZdq{}, ...]}
\PY{c+c1}{\PYZsh{} ...}
\PY{c+c1}{\PYZsh{} For each categorical variable, drop\PYZus{}first=True sorts the dummy columns alphanumerically, than drops the first one}
\PY{c+c1}{\PYZsh{} to avoid the dummy variable trap (perfect multicollinearity among the dummies). Numerical columns are left unchanged}
\PY{n}{X} \PY{o}{=} \PY{n}{pd}\PY{o}{.}\PY{n}{get\PYZus{}dummies}\PY{p}{(}\PY{n}{X}\PY{p}{,} \PY{n}{drop\PYZus{}first}\PY{o}{=}\PY{k+kc}{True}\PY{p}{)}

\PY{c+c1}{\PYZsh{} Force all columns of X and y to be numeric}
\PY{c+c1}{\PYZsh{} errors=\PYZdq{}coerce\PYZdq{} converts anything that cannot be converted to a number}
\PY{c+c1}{\PYZsh{} (weird string, empty space, symbol) into NaN}
\PY{n}{X} \PY{o}{=} \PY{n}{X}\PY{o}{.}\PY{n}{apply}\PY{p}{(}\PY{n}{pd}\PY{o}{.}\PY{n}{to\PYZus{}numeric}\PY{p}{,} \PY{n}{errors}\PY{o}{=}\PY{l+s+s2}{\PYZdq{}}\PY{l+s+s2}{coerce}\PY{l+s+s2}{\PYZdq{}}\PY{p}{)}
\PY{n}{y} \PY{o}{=} \PY{n}{pd}\PY{o}{.}\PY{n}{to\PYZus{}numeric}\PY{p}{(}\PY{n}{y}\PY{p}{,} \PY{n}{errors}\PY{o}{=}\PY{l+s+s2}{\PYZdq{}}\PY{l+s+s2}{coerce}\PY{l+s+s2}{\PYZdq{}}\PY{p}{)}

\PY{c+c1}{\PYZsh{} Combine the y column and all X columns into a single DataFrame}
\PY{c+c1}{\PYZsh{} Then use dropna() to remove all rows that contain at least one NaN in any column.}
\PY{c+c1}{\PYZsh{} e.g., if \PYZdq{}SalePrice\PYZdq{}, \PYZdq{}Gr Liv Area\PYZdq{}, one of the dummy columns is missing, or a string was coerced to NaN}
\PY{c+c1}{\PYZsh{} | y  | x1 | x2 | ... |}
\PY{c+c1}{\PYZsh{} |\PYZhy{}\PYZhy{}\PYZhy{}\PYZhy{}|\PYZhy{}\PYZhy{}\PYZhy{}\PYZhy{}|\PYZhy{}\PYZhy{}\PYZhy{}\PYZhy{}|\PYZhy{}\PYZhy{}\PYZhy{}\PYZhy{}\PYZhy{}|}
\PY{c+c1}{\PYZsh{} | ** | ** | ** | ... |}
\PY{c+c1}{\PYZsh{} | ** | ** | ** | ... |}
\PY{n}{data} \PY{o}{=} \PY{n}{pd}\PY{o}{.}\PY{n}{concat}\PY{p}{(}\PY{p}{[}\PY{n}{y}\PY{p}{,} \PY{n}{X}\PY{p}{]}\PY{p}{,} \PY{n}{axis}\PY{o}{=}\PY{l+m+mi}{1}\PY{p}{)}\PY{o}{.}\PY{n}{dropna}\PY{p}{(}\PY{p}{)}

\PY{c+c1}{\PYZsh{} Take the SalePrice column (y) from the cleaned DataFrame and convert it to a NumPy float array \PYZhy{}\PYZgt{} y\PYZus{}clean: n*1 vector}
\PY{c+c1}{\PYZsh{} Take all remaining columns and convert them to a NumPy float array                             \PYZhy{}\PYZgt{} X\PYZus{}clean: n*k matrix}
\PY{n}{y\PYZus{}clean} \PY{o}{=} \PY{n}{data}\PY{p}{[}\PY{l+s+s2}{\PYZdq{}}\PY{l+s+s2}{SalePrice}\PY{l+s+s2}{\PYZdq{}}\PY{p}{]}\PY{o}{.}\PY{n}{to\PYZus{}numpy}\PY{p}{(}\PY{n}{dtype}\PY{o}{=}\PY{n+nb}{float}\PY{p}{)}
\PY{n}{X\PYZus{}clean} \PY{o}{=} \PY{n}{data}\PY{o}{.}\PY{n}{drop}\PY{p}{(}\PY{n}{columns}\PY{o}{=}\PY{p}{[}\PY{l+s+s2}{\PYZdq{}}\PY{l+s+s2}{SalePrice}\PY{l+s+s2}{\PYZdq{}}\PY{p}{]}\PY{p}{)}\PY{o}{.}\PY{n}{to\PYZus{}numpy}\PY{p}{(}\PY{n}{dtype}\PY{o}{=}\PY{n+nb}{float}\PY{p}{)}

\PY{c+c1}{\PYZsh{} Add a column of ones to the matrix (the intercept b\PYZus{}0)}
\PY{n}{X\PYZus{}clean} \PY{o}{=} \PY{n}{sm}\PY{o}{.}\PY{n}{add\PYZus{}constant}\PY{p}{(}\PY{n}{X\PYZus{}clean}\PY{p}{)}

\PY{c+c1}{\PYZsh{} Create the OLS model object and call fit() to estimate the coefficients (b̂) by minimizing the sum of squared residuals:}
\PY{c+c1}{\PYZsh{} given y = X*b + e, it finds the b̂ that make the residuals as small as possible}
\PY{n}{model} \PY{o}{=} \PY{n}{sm}\PY{o}{.}\PY{n}{OLS}\PY{p}{(}\PY{n}{y\PYZus{}clean}\PY{p}{,} \PY{n}{X\PYZus{}clean}\PY{p}{)}\PY{o}{.}\PY{n}{fit}\PY{p}{(}\PY{p}{)}
\PY{n+nb}{print}\PY{p}{(}\PY{n}{model}\PY{o}{.}\PY{n}{summary}\PY{p}{(}\PY{p}{)}\PY{p}{)}
\end{Verbatim}
\end{tcolorbox}

    \begin{Verbatim}[commandchars=\\\{\}]
                            OLS Regression Results
==============================================================================
Dep. Variable:                      y   R-squared:                       0.855
Model:                            OLS   Adj. R-squared:                  0.852
Method:                 Least Squares   F-statistic:                     280.9
Date:                Wed, 19 Nov 2025   Prob (F-statistic):               0.00
Time:                        20:54:57   Log-Likelihood:                -32513.
No. Observations:                2770   AIC:                         6.514e+04
Df Residuals:                    2712   BIC:                         6.549e+04
Df Model:                          57
Covariance Type:            nonrobust
==============================================================================
                 coef    std err          t      P>|t|      [0.025      0.975]
------------------------------------------------------------------------------
const      -8.876e+05   1.32e+05     -6.724      0.000   -1.15e+06   -6.29e+05
x1            44.3995      2.027     21.906      0.000      40.425      48.374
x2            -3.9192      3.104     -1.263      0.207     -10.006       2.167
x3            23.4093      2.665      8.784      0.000      18.183      28.635
x4             0.6152      0.085      7.212      0.000       0.448       0.782
x5         -3756.4845   1658.574     -2.265      0.024   -7008.682    -504.287
x6            16.5662      6.695      2.475      0.013       3.439      29.693
x7          7228.4564   1931.315      3.743      0.000    3441.459     1.1e+04
x8           -39.5767     47.771     -0.828      0.407    -133.249      54.095
x9           247.2766     57.053      4.334      0.000     135.404     359.149
x10          259.1812     44.615      5.809      0.000     171.698     346.664
x11         1.261e+04    820.520     15.370      0.000     1.1e+04    1.42e+04
x12        -3.551e+04   5801.478     -6.120      0.000   -4.69e+04   -2.41e+04
x13        -2.923e+04   3179.782     -9.192      0.000   -3.55e+04    -2.3e+04
x14        -5.268e+04   3.14e+04     -1.678      0.094   -1.14e+05    8889.597
x15        -3.553e+04   3552.403    -10.001      0.000   -4.25e+04   -2.86e+04
x16         -3.79e+04   8692.842     -4.360      0.000   -5.49e+04   -2.09e+04
x17        -2.849e+04   4109.779     -6.931      0.000   -3.65e+04   -2.04e+04
x18        -2.825e+04   4651.880     -6.073      0.000   -3.74e+04   -1.91e+04
x19        -2.125e+04   4794.607     -4.433      0.000   -3.07e+04   -1.19e+04
x20        -2.251e+04   2492.245     -9.033      0.000   -2.74e+04   -1.76e+04
x21        -2909.7269   2.36e+04     -0.123      0.902   -4.93e+04    4.35e+04
x22         -1.96e+04   2978.602     -6.581      0.000   -2.54e+04   -1.38e+04
x23         2288.6955   1.19e+04      0.192      0.848   -2.11e+04    2.56e+04
x24        -7326.1849   9249.664     -0.792      0.428   -2.55e+04    1.08e+04
x25         1.098e+04   7863.407      1.397      0.163   -4436.016    2.64e+04
x26         2.192e+04   7928.231      2.765      0.006    6374.661    3.75e+04
x27         8528.3242   6233.377      1.368      0.171   -3694.325    2.08e+04
x28         3.186e+04   7146.519      4.459      0.000    1.79e+04    4.59e+04
x29        -8111.6946   6852.156     -1.184      0.237   -2.15e+04    5324.281
x30         4281.1987   6486.531      0.660      0.509   -8437.844     1.7e+04
x31         1.275e+04   1.25e+04      1.022      0.307   -1.17e+04    3.72e+04
x32         1.129e+05   2.28e+04      4.962      0.000    6.83e+04    1.57e+05
x33         7997.7135   8736.543      0.915      0.360   -9133.242    2.51e+04
x34        -3216.6788    3.2e+04     -0.101      0.920    -6.6e+04    5.95e+04
x35         2471.7518   9517.391      0.260      0.795   -1.62e+04    2.11e+04
x36         2205.6500   6858.340      0.322      0.748   -1.12e+04    1.57e+04
x37         3331.9295   6567.072      0.507      0.612   -9545.042    1.62e+04
x38        -5548.4113   8930.430     -0.621      0.534   -2.31e+04     1.2e+04
x39         3487.8154   6725.315      0.519      0.604   -9699.446    1.67e+04
x40         5.949e+04   7149.219      8.321      0.000    4.55e+04    7.35e+04
x41         3.104e+04   6570.613      4.724      0.000    1.82e+04    4.39e+04
x42         1234.4781   7905.291      0.156      0.876   -1.43e+04    1.67e+04
x43          868.9178   8469.032      0.103      0.918   -1.57e+04    1.75e+04
x44         3828.3917   6851.807      0.559      0.576   -9606.899    1.73e+04
x45          732.8101   6622.674      0.111      0.912   -1.23e+04    1.37e+04
x46         1.747e+04   7563.024      2.310      0.021    2640.863    3.23e+04
x47         5.065e+04   7397.165      6.847      0.000    3.61e+04    6.52e+04
x48         1.653e+04   7004.926      2.360      0.018    2793.712    3.03e+04
x49         2.556e+04   8697.728      2.939      0.003    8509.290    4.26e+04
x50         7187.1732   2.39e+04      0.301      0.764   -3.97e+04     5.4e+04
x51         2.269e+04   2.33e+04      0.976      0.329   -2.29e+04    6.83e+04
x52         1.445e+04   3.35e+04      0.431      0.667   -5.13e+04    8.02e+04
x53         1.963e+04   2.36e+04      0.831      0.406   -2.67e+04    6.59e+04
x54         2.938e+04   2.26e+04      1.299      0.194    -1.5e+04    7.37e+04
x55         1.912e+04   2.26e+04      0.845      0.398   -2.52e+04    6.35e+04
x56        -3.536e+04    3.1e+04     -1.141      0.254   -9.61e+04    2.54e+04
x57        -2.675e+04   2.53e+04     -1.058      0.290   -7.63e+04    2.28e+04
==============================================================================
Omnibus:                     1640.604   Durbin-Watson:                   1.703
Prob(Omnibus):                  0.000   Jarque-Bera (JB):           230338.424
Skew:                          -1.805   Prob(JB):                         0.00
Kurtosis:                      47.527   Cond. No.                     3.08e+06
==============================================================================

Notes:
[1] Standard Errors assume that the covariance matrix of the errors is correctly
specified.
[2] The condition number is large, 3.08e+06. This might indicate that there are
strong multicollinearity or other numerical problems.
    \end{Verbatim}

    \hypertarget{interpretation-of-the-regression-output}{%
\subsection{Interpretation of the regression
output}\label{interpretation-of-the-regression-output}}

The OLS regression is estimated on 2,770 observations.\\
The dependent variable is \texttt{SalePrice}, while the regressors
include size variables, quality indicators, age/renovation variables and
a set of dummies for neighbourhood, zoning and utilities.

\hypertarget{overall-fit-of-the-model}{%
\subsubsection{Overall fit of the
model}\label{overall-fit-of-the-model}}

The model displays a relatively high goodness of fit:

\begin{itemize}
\tightlist
\item
  \texttt{R-squared} = 0.855\\
\item
  \texttt{Adjusted\ R-squared} = 0.852
\end{itemize}

This means that about 85\% of the variation in house prices in the
sample is explained by the regressors included in the model.

The \texttt{F-statistic} is 280.9 with a \texttt{p-value} essentially
equal to zero, so the null hypothesis that all slope coefficients are
jointly equal to zero is strongly rejected. Taken together, the
explanatory variables have a statistically significant impact on
\texttt{SalePrice}.

\hypertarget{interpretation-of-the-main-coefficients}{%
\subsubsection{Interpretation of the main
coefficients}\label{interpretation-of-the-main-coefficients}}

The first group of coefficients (x1--x11) corresponds to the ``core''
numeric regressors (size, age, quality):

\begin{itemize}
\tightlist
\item
  \texttt{x1} \(\approx 44.4\) (\texttt{Gr\ Liv\ Area}) is positive and
  highly significant Interpreting it as the coefficient on above-ground
  living area, it suggests that, ceteris paribus, an additional square
  foot of living area increases the sale price by about 44 dollars.
\item
  \texttt{x3} \(\approx 23.4\) (\texttt{Total\ Bsmt\ SF}) is also
  positive and highly significant, indicating that extra basement area
  is rewarded in the market (around 23 dollars per square foot)
\item
  \texttt{x4} \(\approx 0.62\) (\texttt{Lot\ Area}) is positive and
  significant: larger lots are associated with higher prices, although
  the marginal effect per square foot is much smaller than for interior
  living space or basement area
\item
  \texttt{x5} \(\approx –3,756\) (\texttt{Full\ Bath}) is negative and
  statistically significant at the 5\% level.\\
  This is somewhat counter-intuitive, but it can be explained by
  multicollinearity: once we condition on total size and quality, adding
  one more full bathroom at fixed size may capture houses with a
  different layout rather than genuinely higher quality
\item
  \texttt{x6} \(\approx 16.6\) (\texttt{Garage\ Area}) and \texttt{x7}
  \(\approx 7,228\) (\texttt{Garage\ Cars}) are both positive and
  significant, meaning that larger garages and the capacity to park more
  cars are rewarded by the market
\item
  \texttt{x8} (\texttt{Garage\ Yr\ Blt}) is not statistically
  significant (p \(\approx 0.41\)), suggesting that, after controlling
  for the other variables, the construction year of the garage does not
  have a clear marginal effect
\item
  \texttt{x9} \(\approx 247.3\) (\texttt{Year\ Built}) and \texttt{x10}
  \(\approx 259.2\) (\texttt{Year\ Remod/Add}) are positive and highly
  significant: newer houses and more recently renovated properties tend
  to sell at higher prices, in line with the idea that buyers pay a
  premium for newer structures and recent improvements
\item
  \texttt{x11} \(\approx 12,610\) (\texttt{Overall\ Qual}) is positive
  and extremely significant A one-point increase in the overall quality
  rating is associated with roughly 12--13 thousand dollars more in sale
  price, everything else being equal.\\
  This confirms the strong role of quality emphasised in the hedonic
  pricing literature.
\end{itemize}

The next coefficients (\texttt{x12}, \texttt{x13},
\texttt{x15}--\texttt{x20}, \texttt{x22}, etc.) correspond to dummy
variables created from \texttt{Kitchen\ Qual}, \texttt{Exter\ Qual} and
\texttt{Bsmt\ Qual}. Most of these are negative and highly significant,
because the baseline category is the highest quality level, and the
dummies measure the price discount associated with having a lower
quality relative to that baseline.\\
In other words, houses with worse kitchen, exterior or basement quality
sell at systematically lower prices.

The block of coefficients starting around \texttt{x28}--\texttt{x41}
corresponds mostly to \texttt{Neighborhood} dummies.\\
Some neighbourhoods show large positive and significant coefficients
(for example, \texttt{x32}, \texttt{x40}, \texttt{x41}, \texttt{x47},
\texttt{x48}, \texttt{x49}), indicating that these areas command
sizeable price premia relative to the omitted reference neighbourhood.\\
Other neighbourhood coefficients are not statistically significant,
suggesting that, after controlling for structural and quality
characteristics, their effect on price is not clearly distinguishable
from zero.

Finally, the last coefficients (approximately
\texttt{x50}--\texttt{x57}) represent dummies for \texttt{MS\ Zoning}
and \texttt{Utilities}.\\
Most of them are not statistically significant (p-values well above
0.05), which indicates that, in this specification, zoning and utilities
categories do not add much explanatory power once size, quality and
neighbourhood are already in the model.

    \hypertarget{covariance-matrix}{%
\subsection{Covariance matrix}\label{covariance-matrix}}

    \begin{tcolorbox}[breakable, size=fbox, boxrule=1pt, pad at break*=1mm,colback=cellbackground, colframe=cellborder]
\prompt{In}{incolor}{4}{\boxspacing}
\begin{Verbatim}[commandchars=\\\{\}]
\PY{k+kn}{import}\PY{+w}{ }\PY{n+nn}{numpy}\PY{+w}{ }\PY{k}{as}\PY{+w}{ }\PY{n+nn}{np}
\PY{k+kn}{import}\PY{+w}{ }\PY{n+nn}{matplotlib}\PY{n+nn}{.}\PY{n+nn}{pyplot}\PY{+w}{ }\PY{k}{as}\PY{+w}{ }\PY{n+nn}{plt}
\PY{k+kn}{import}\PY{+w}{ }\PY{n+nn}{seaborn}\PY{+w}{ }\PY{k}{as}\PY{+w}{ }\PY{n+nn}{sns}

\PY{c+c1}{\PYZsh{} select only the relevant numeric columns}
\PY{n}{num\PYZus{}vars} \PY{o}{=} \PY{p}{[}
    \PY{l+s+s2}{\PYZdq{}}\PY{l+s+s2}{Gr Liv Area}\PY{l+s+s2}{\PYZdq{}}\PY{p}{,}
    \PY{l+s+s2}{\PYZdq{}}\PY{l+s+s2}{1st Flr SF}\PY{l+s+s2}{\PYZdq{}}\PY{p}{,}
    \PY{l+s+s2}{\PYZdq{}}\PY{l+s+s2}{Total Bsmt SF}\PY{l+s+s2}{\PYZdq{}}\PY{p}{,}
    \PY{l+s+s2}{\PYZdq{}}\PY{l+s+s2}{Lot Area}\PY{l+s+s2}{\PYZdq{}}\PY{p}{,}
    \PY{l+s+s2}{\PYZdq{}}\PY{l+s+s2}{Full Bath}\PY{l+s+s2}{\PYZdq{}}\PY{p}{,}
    \PY{l+s+s2}{\PYZdq{}}\PY{l+s+s2}{Garage Area}\PY{l+s+s2}{\PYZdq{}}\PY{p}{,}
    \PY{l+s+s2}{\PYZdq{}}\PY{l+s+s2}{Garage Cars}\PY{l+s+s2}{\PYZdq{}}\PY{p}{,}
    \PY{l+s+s2}{\PYZdq{}}\PY{l+s+s2}{Garage Yr Blt}\PY{l+s+s2}{\PYZdq{}}\PY{p}{,}
    \PY{l+s+s2}{\PYZdq{}}\PY{l+s+s2}{Year Built}\PY{l+s+s2}{\PYZdq{}}\PY{p}{,}
    \PY{l+s+s2}{\PYZdq{}}\PY{l+s+s2}{Year Remod/Add}\PY{l+s+s2}{\PYZdq{}}\PY{p}{,}
\PY{p}{]}

\PY{n}{corr} \PY{o}{=} \PY{n}{df}\PY{p}{[}\PY{n}{num\PYZus{}vars}\PY{p}{]}\PY{o}{.}\PY{n}{corr}\PY{p}{(}\PY{p}{)}

\PY{n}{plt}\PY{o}{.}\PY{n}{figure}\PY{p}{(}\PY{n}{figsize}\PY{o}{=}\PY{p}{(}\PY{l+m+mi}{8}\PY{p}{,} \PY{l+m+mi}{6}\PY{p}{)}\PY{p}{)}
\PY{n}{sns}\PY{o}{.}\PY{n}{heatmap}\PY{p}{(}\PY{n}{corr}\PY{p}{,} \PY{n}{annot}\PY{o}{=}\PY{k+kc}{True}\PY{p}{,} \PY{n}{fmt}\PY{o}{=}\PY{l+s+s2}{\PYZdq{}}\PY{l+s+s2}{.2f}\PY{l+s+s2}{\PYZdq{}}\PY{p}{,} \PY{n}{cmap}\PY{o}{=}\PY{l+s+s2}{\PYZdq{}}\PY{l+s+s2}{coolwarm}\PY{l+s+s2}{\PYZdq{}}\PY{p}{)}

\PY{c+c1}{\PYZsh{} colormap dafault}
\PY{c+c1}{\PYZsh{} im = plt.imshow(corr, vmin=\PYZhy{}1, vmax=1)}

\PY{n}{plt}\PY{o}{.}\PY{n}{xticks}\PY{p}{(}\PY{n}{np}\PY{o}{.}\PY{n}{arange}\PY{p}{(}\PY{n+nb}{len}\PY{p}{(}\PY{n}{num\PYZus{}vars}\PY{p}{)}\PY{p}{)}\PY{p}{,} \PY{n}{num\PYZus{}vars}\PY{p}{,} \PY{n}{rotation}\PY{o}{=}\PY{l+m+mi}{90}\PY{p}{)}
\PY{n}{plt}\PY{o}{.}\PY{n}{yticks}\PY{p}{(}\PY{n}{np}\PY{o}{.}\PY{n}{arange}\PY{p}{(}\PY{n+nb}{len}\PY{p}{(}\PY{n}{num\PYZus{}vars}\PY{p}{)}\PY{p}{)}\PY{p}{,} \PY{n}{num\PYZus{}vars}\PY{p}{)}

\PY{c+c1}{\PYZsh{} colorbar (to be used with the colormap)}
\PY{c+c1}{\PYZsh{} plt.colorbar(im, fraction=0.046, pad=0.04)}

\PY{n}{plt}\PY{o}{.}\PY{n}{title}\PY{p}{(}\PY{l+s+s2}{\PYZdq{}}\PY{l+s+s2}{Correlation matrix}\PY{l+s+s2}{\PYZdq{}}\PY{p}{)}
\PY{n}{plt}\PY{o}{.}\PY{n}{tight\PYZus{}layout}\PY{p}{(}\PY{p}{)}
\PY{n}{plt}\PY{o}{.}\PY{n}{show}\PY{p}{(}\PY{p}{)}
\end{Verbatim}
\end{tcolorbox}

    \begin{center}
    \adjustimage{max size={0.9\linewidth}{0.9\paperheight}}{Group10 AmesHousing_files/Group10 AmesHousing_23_0.png}
    \end{center}
    { \hspace*{\fill} \\}
    
    \bibliographystyle{plain}
\bibliography{references}


    % Add a bibliography block to the postdoc
    
    
    
\end{document}
