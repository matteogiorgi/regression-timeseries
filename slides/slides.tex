\documentclass{beamer}

\usetheme{Warsaw} % Madrid, CambridgeUS, Warsaw
% \setbeamercolor{title in head/foot}{bg=lightgray,fg=black}
\setbeamersize{text margin left=0.7cm, text margin right=0.7cm}
\setbeamertemplate{navigation symbols}{}
\usepackage[utf8]{inputenc}
\usepackage{booktabs}
\usepackage{graphicx}
\usepackage{listings}
\usepackage{xcolor}
\usepackage{hyperref}
\hypersetup{
    colorlinks=true,
    urlcolor=blue!60!black,
    linkcolor=black,
    citecolor=black
}

\definecolor{codegray}{rgb}{0.95,0.95,0.95}
\definecolor{commentgreen}{rgb}{0,0.6,0}
\definecolor{keywordblue}{rgb}{0.2,0.2,0.8}
\definecolor{stringred}{rgb}{0.7,0,0}

\lstdefinestyle{pythonstyle}{
    backgroundcolor=\color{codegray},
    commentstyle=\color{commentgreen}\itshape,
    keywordstyle=\color{keywordblue}\bfseries,
    stringstyle=\color{stringred},
    basicstyle=\ttfamily\scriptsize,
    showstringspaces=false,
    breaklines=true,
    frame=single,
    rulecolor=\color{gray},
    captionpos=b,
    language=Python
}

\title[AI Adoption \& Layoffs]{AI Adoption and Layoffs}
\author{\textbf{Group 10}}
\date{\today}

\begin{document}

%=================================================
\begin{frame}
  \titlepage
\end{frame}
%=================================================

\begin{frame}{Research Question and Model Definition}
%   \begin{itemize}
%     \item Does a higher level of AI adoption in firms correlate with a higher number of layoffs?
%     \item We build a simple cross-sectional linear regression with synthetic data using the following law:
%     \[
%       \text{Layoffs}_i = \beta_0 + \beta_1 \text{AI\_Adoption}_i + \varepsilon_i
%     \]
%     \begin{itemize}
%         \item we simulate 50 firms total
%         \item $\text{AI\_Adoption}_i$ represents uniform score in \([0,100]\)
%         \item $\text{Layoffs}_i$ represents the baseline $20$ + $1.5 \times$ AI adoption + noise $\mathcal{N}(0,15)$
%     \item Noise added to avoid a perfect linear cloud.
%     \end{itemize}
%   \end{itemize}
\begin{itemize}
  \item Does a higher level of AI adoption in firms correlate with a higher number of layoffs?
  \item We build a simple cross-sectional linear regression with synthetic data using the following model:
  \[
    \text{Layoffs}_i = \beta_0 + \beta_1 \, \text{AI\_Adoption}_i + \varepsilon_i
  \]
  \item \textbf{AI\_Adoption$_i$}: represents the degree of AI integration within firm $i$  
        (e.g., automation of processes, use of machine learning tools, or AI-driven decision systems), expressed as a percentage
  \item \textbf{Layoffs$_i$}: number of employees laid off by firm $i$ during the year
  \item Simulated dataset: 100 firms
    % \begin{itemize}
    %   \item $\text{AI\_Adoption}_i \sim U(0, 100)$
    %   \item $\text{Layoffs}_i = 20 + 1.5 \times \text{AI\_Adoption}_i + \mathcal{N}(0,15)$
    %   \item Noise added to avoid a perfect linear cloud
    % \end{itemize}
\end{itemize}
\end{frame}
%=================================================

% \begin{frame}{Data Generation (Synthetic)}
%   \begin{itemize}
%     \item 50 firms.
%     \item \textbf{AI\_Adoption}: uniform score in \([0,100]\).
%     \item \textbf{Layoffs}: baseline $20$ + $1.5 \times$ AI adoption + noise $\mathcal{N}(0,15)$.
%     \item Noise added to avoid a perfect linear cloud.
%   \end{itemize}
% \end{frame}
%=================================================

% \begin{frame}{Model Specification}
%   \begin{itemize}
%     \item Dependent variable: \textbf{Layoffs} (number of layoffs in the firm).
%     \item Independent variable: \textbf{AI\_Adoption} (0--100 score).
%     \item Estimation method: OLS.
%     \item Software: Python, \texttt{statsmodels}.
%   \end{itemize}
% \end{frame}
%=================================================

% \begin{frame}{OLS Results (summary)}
%   \small
%   \begin{tabular}{lccc}
%     \toprule
%     & Coef. & Std. Err. & p-value \\
%     \midrule
%     Constant & 21.4503 & 3.639 & 0.000 \\
%     AI\_Adoption & 1.4665 & 0.069 & 0.000 \\
%     \midrule
%     \multicolumn{4}{l}{R-squared = 0.905 \quad Adj. R-squared = 0.903} \\
%     \multicolumn{4}{l}{F-statistic = 455.9 \quad Prob(F) = 3.75e-26} \\
%     \bottomrule
%   \end{tabular}

%   \vspace{0.3cm}
%   \footnotesize
%   Note: standard errors assume correctly specified covariance matrix.
% \end{frame}

%=================================================

\begin{frame}[fragile]{OLS Estimation Code}
In the following slides, we show selected Python code snippets to illustrate 
the main steps of our regression analysis.

\vspace{1.0cm}
\lstset{style=pythonstyle}
\begin{lstlisting}
# Add a constant term to include the intercept in the model
# and define the dependent variable (layoffs)
X = sm.add_constant(df["AI_Adoption"])
y = df["Layoffs"]

# Fit the OLS regression model
# Dand display the regression summary
model = sm.OLS(y, X).fit()
print(model.summary())
\end{lstlisting}

\vspace{0.3cm}
\scriptsize
The complete and reproducible code is available at:
\href{https://github.com/matteogiorgi/regression-timeseries/blob/main/layoff.ipynb}{\textcolor{blue!70!black}{github.com/matteogiorgi/regression-timeseries/.../layoff.ipynb}}
% \texttt{https://github.com/matteogiorgi/regression-timeseries}
\end{frame}

%=================================================

\begin{frame}[fragile]{OLS Model Output}
\scriptsize
\begin{verbatim}
                            OLS Regression Results                            
==============================================================================
Dep. Variable:                Layoffs   R-squared:                       0.908
Model:                            OLS   Adj. R-squared:                  0.907
Method:                 Least Squares   F-statistic:                     968.9
Date:                Fri, 07 Nov 2025   Prob (F-statistic):           1.31e-52
Time:                        11:59:07   Log-Likelihood:                -401.95
No. Observations:                 100   AIC:                             807.9
Df Residuals:                      98   BIC:                             813.1
Df Model:                           1                                         
Covariance Type:            nonrobust                                         
===============================================================================
                  coef    std err          t      P>|t|      [0.025      0.975]
-------------------------------------------------------------------------------
const          23.2264      2.554      9.093      0.000      18.158      28.295
AI_Adoption     1.4310      0.046     31.127      0.000       1.340       1.522
==============================================================================
Omnibus:                        0.900   Durbin-Watson:                   2.285
Prob(Omnibus):                  0.638   Jarque-Bera (JB):                0.808
Skew:                           0.217   Prob(JB):                        0.668
Kurtosis:                       2.929   Cond. No.                         104.
==============================================================================
\end{verbatim}
\end{frame}

%=================================================

\begin{frame}[fragile]{Plot Fitted Model}
\lstset{style=pythonstyle}
\begin{lstlisting}
# Predict fitted values based on the estimated model
y_pred = model.predict(X)

# Create scatter plot (observed data)
# and add regression line (predicted values)
plt.figure(figsize=(7, 5))
plt.scatter(
    df["AI_Adoption"],
    df["Layoffs"],
    label="Observed data"
)
plt.plot(
    df["AI_Adoption"],
    y_pred,
    color="red",
    label="Fitted line"
)
plt.xlabel("AI Adoption Level (%)")
plt.ylabel("Number of Layoffs")
plt.title("Relationship between AI Adoption and Layoffs")
plt.legend()
plt.tight_layout()
plt.show()
\end{lstlisting}
\end{frame}
%=================================================

\begin{frame}{Scatter Plot and Fitted Line}
  \begin{center}
    \includegraphics[width=0.9\linewidth]{image.png}
  \end{center}
  \footnotesize
%   The fitted line closely follows the data, confirming the positive linear relationship.
\end{frame}

%=================================================

% \begin{frame}{Interpretation}
%   \begin{itemize}
%     \item The coefficient on AI adoption is \textbf{1.4310} and it is highly significant.
%     \item This means: for a 1-point increase in the AI adoption score, layoffs increase on average by about 1.43 units.
%     \item Very high $R^2 = 0.908$ because the data were constructed to reflect a strong positive relationship.
%     \item In real data we would expect more noise and lower $R^2$.
%   \end{itemize}
% \end{frame}

%=================================================

% \begin{frame}{Conclusions and AI usage}
%   \begin{itemize}
%     \item Our synthetic example shows a clear positive correlation between AI adoption and layoffs.
%     \item The result is statistically strong (very small p-value).
%     \item For an applied project, we would:
%       \begin{itemize}
%         \item collect real firm-level data,
%         \item add control variables (firm size, sector, profitability),
%         \item test robustness.
%       \end{itemize}
%   \end{itemize}
% \end{frame}

%=================================================

\begin{frame}{Interpretation of the Regression Results}
  \begin{itemize}
    \item The estimated coefficient for \textbf{AI\_Adoption} is \textbf{1.43}, statistically significant at the 1\% level ($p < 0.001$).
    \item Interpretation: for each one-point percentage increase in the AI adoption index, the number of layoffs increases on average by about 1.43 employees.
    \item The intercept ($\beta_0 \approx 23.2$) indicates the expected number of layoffs for firms with no AI adoption at all.
    \item The $R^2 = 0.91$ shows that roughly \textbf{91\% of the variation in layoffs} across firms is explained by differences in AI adoption levels.
    \item These results suggest a strong positive association between automation intensity and workforce reduction — consistent with the hypothesis that higher AI adoption may substitute part of human labor.
  \end{itemize}
\end{frame}

%=================================================

\begin{frame}{Our use of LLM}
  \begin{itemize}
    \item We employed \textit{ChatGPT} from \textit{OpenAI} as AI peer-wise support tool during the preparation of this work
    \item Specifically, the model was asked to:
      \begin{itemize}
        \item generate a set of \textbf{synthetic yet realistic data} for 100 firms,  
              including a measure of AI adoption (\texttt{AI\_Adoption}) and the corresponding number of layoffs (\texttt{Layoffs})
        \item propose and explain a \textbf{linear regression model} where:
          \begin{itemize}
            \item the independent variable is the degree of AI adoption
            \item the dependent variable is the number of layoffs per firm
          \end{itemize}
      \end{itemize}
    \item All results were than independently reviewed and verified by our group, together with the interpretations provided
  \end{itemize}
\end{frame}

%=================================================

\begin{frame}{Discussion and Takeaways}
  \begin{itemize}
    \item The model highlights how AI-driven automation could lead to higher layoffs, at least in the short run.
    \item However, this synthetic example only captures a simplified linear relationship:
      \begin{itemize}
        \item Real-world dynamics may depend on sector, firm size, and type of AI integration.
        \item In some industries, AI adoption may create new roles (data analysis, system maintenance) rather than destroy jobs.
      \end{itemize}
    \item Future empirical work should:
      \begin{itemize}
        \item use panel data or time-series evidence to distinguish correlation from causality;
        \item include control variables (e.g., productivity, profitability, R\&D intensity);
        \item test for nonlinear or threshold effects in AI adoption.
      \end{itemize}
    \item Overall, the analysis illustrates how regression methods can be used to quantify economic effects of technological innovation.
  \end{itemize}
\end{frame}

%=================================================

\begin{frame}{}
  \begin{center}
    % \vspace{1.5cm}
    % \Large
    This work was conceived and developed by:\\[0.3cm]
    \Large\textit{Mattia Zanin, Matteo Giorgi, Enrico Zanello,\\Klea Kule, Luca Lo Buono}\\[1.0cm]
    \normalsize
    University of Padova\\2025/2026
    % \textit{Regression and Time Series\\University of Padova}%\\[0.2cm]
    % Academic Year 2025/2026
  \end{center}
\end{frame}

\end{document}
