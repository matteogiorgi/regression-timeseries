\documentclass{beamer}

\usetheme{Madrid} % va bene anche CambridgeUS o Warsaw
\usepackage[utf8]{inputenc}
\usepackage{booktabs}
\usepackage{graphicx}

\title[AI Adoption \& Layoffs]{AI Adoption and Layoffs}
\author{Your Team}
\date{\today}

\begin{document}

%=================================================
\begin{frame}
  \titlepage
\end{frame}
%=================================================

\begin{frame}{Research Question and Model Definition}
%   \begin{itemize}
%     \item Does a higher level of AI adoption in firms correlate with a higher number of layoffs?
%     \item We build a simple cross-sectional linear regression with synthetic data using the following law:
%     \[
%       \text{Layoffs}_i = \beta_0 + \beta_1 \text{AI\_Adoption}_i + \varepsilon_i
%     \]
%     \begin{itemize}
%         \item we simulate 50 firms total
%         \item $\text{AI\_Adoption}_i$ represents uniform score in \([0,100]\)
%         \item $\text{Layoffs}_i$ represents the baseline $20$ + $1.5 \times$ AI adoption + noise $\mathcal{N}(0,15)$
%     \item Noise added to avoid a perfect linear cloud.
%     \end{itemize}
%   \end{itemize}
\begin{itemize}
    \item Does a higher level of AI adoption in firms correlate with a higher number of layoffs?
  \item We build a simple cross-sectional linear regression with synthetic data using the following model:
  \[
    \text{Layoffs}_i = \beta_0 + \beta_1 \, \text{AI\_Adoption}_i + \varepsilon_i
  \]
  \item \textbf{AI\_Adoption$_i$}: represents the degree of AI integration within firm $i$  
        (e.g., automation of processes, use of machine learning tools, or AI-driven decision systems), scaled between 0 and 100
  \item \textbf{Layoffs$_i$}: number of employees laid off by firm $i$ during the year
  \item Simulated dataset: 100 firms
    % \begin{itemize}
    %   \item $\text{AI\_Adoption}_i \sim U(0, 100)$
    %   \item $\text{Layoffs}_i = 20 + 1.5 \times \text{AI\_Adoption}_i + \mathcal{N}(0,15)$
    %   \item Noise added to avoid a perfect linear cloud
    % \end{itemize}
\end{itemize}
\end{frame}
%=================================================

% \begin{frame}{Data Generation (Synthetic)}
%   \begin{itemize}
%     \item 50 firms.
%     \item \textbf{AI\_Adoption}: uniform score in \([0,100]\).
%     \item \textbf{Layoffs}: baseline $20$ + $1.5 \times$ AI adoption + noise $\mathcal{N}(0,15)$.
%     \item Noise added to avoid a perfect linear cloud.
%   \end{itemize}
% \end{frame}
%=================================================

% \begin{frame}{Model Specification}
%   \begin{itemize}
%     \item Dependent variable: \textbf{Layoffs} (number of layoffs in the firm).
%     \item Independent variable: \textbf{AI\_Adoption} (0--100 score).
%     \item Estimation method: OLS.
%     \item Software: Python, \texttt{statsmodels}.
%   \end{itemize}
% \end{frame}
%=================================================

% \begin{frame}{OLS Results (summary)}
%   \small
%   \begin{tabular}{lccc}
%     \toprule
%     & Coef. & Std. Err. & p-value \\
%     \midrule
%     Constant & 21.4503 & 3.639 & 0.000 \\
%     AI\_Adoption & 1.4665 & 0.069 & 0.000 \\
%     \midrule
%     \multicolumn{4}{l}{R-squared = 0.905 \quad Adj. R-squared = 0.903} \\
%     \multicolumn{4}{l}{F-statistic = 455.9 \quad Prob(F) = 3.75e-26} \\
%     \bottomrule
%   \end{tabular}

%   \vspace{0.3cm}
%   \footnotesize
%   Note: standard errors assume correctly specified covariance matrix.
% \end{frame}
%=================================================

\begin{frame}{Scatter Plot and Fitted Line}
  \begin{center}
    \includegraphics[width=0.9\linewidth]{image.png}
  \end{center}
  \footnotesize
  The fitted line closely follows the data, confirming the positive linear relationship.
\end{frame}
%=================================================

\begin{frame}[fragile]{OLS Model Output}
\scriptsize
\begin{verbatim}
                            OLS Regression Results                            
==============================================================================
Dep. Variable:                Layoffs   R-squared:                       0.908
Model:                            OLS   Adj. R-squared:                  0.907
Method:                 Least Squares   F-statistic:                     968.9
Date:                Fri, 07 Nov 2025   Prob (F-statistic):           1.31e-52
Time:                        11:59:07   Log-Likelihood:                -401.95
No. Observations:                 100   AIC:                             807.9
Df Residuals:                      98   BIC:                             813.1
Df Model:                           1                                         
Covariance Type:            nonrobust                                         
===============================================================================
                  coef    std err          t      P>|t|      [0.025      0.975]
-------------------------------------------------------------------------------
const          23.2264      2.554      9.093      0.000      18.158      28.295
AI_Adoption     1.4310      0.046     31.127      0.000       1.340       1.522
==============================================================================
Omnibus:                        0.900   Durbin-Watson:                   2.285
Prob(Omnibus):                  0.638   Jarque-Bera (JB):                0.808
Skew:                           0.217   Prob(JB):                        0.668
Kurtosis:                       2.929   Cond. No.                         104.
==============================================================================
\end{verbatim}
\end{frame}

%=================================================

\begin{frame}{Interpretation}
  \begin{itemize}
    \item The coefficient on AI adoption is \textbf{1.4665} and it is highly significant.
    \item This means: for a 1-point increase in the AI adoption score, layoffs increase on average by about 1.47 units.
    \item Very high $R^2 = 0.905$ because the data were constructed to reflect a strong positive relationship.
    \item In real data we would expect more noise and lower $R^2$.
  \end{itemize}
\end{frame}

%=================================================

\begin{frame}{Conclusions and AI usage}
  \begin{itemize}
    \item Our synthetic example shows a clear positive correlation between AI adoption and layoffs.
    \item The result is statistically strong (very small p-value).
    \item For an applied project, we would:
      \begin{itemize}
        \item collect real firm-level data,
        \item add control variables (firm size, sector, profitability),
        \item test robustness.
      \end{itemize}
  \end{itemize}
\end{frame}
%=================================================

\end{document}
